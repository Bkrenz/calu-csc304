\documentclass[12pt]{article}

\usepackage[english]{babel}

\usepackage{amsmath}

\usepackage{graphicx}
\usepackage{tabularx}
\usepackage{enumitem}

\usepackage{mathptmx}
\usepackage{setspace}
\doublespacing 

% Geometry 
\usepackage{geometry}
\geometry{letterpaper, left=1in, top=1in, right=1in, bottom=1in}

% Add vertical spacing to tables
\renewcommand{\arraystretch}{2}

% Macros
\newcommand{\definition}[1]{\underline{\textbf{#1}}}

% Begin Document
\begin{document}

Robert Krency - Quanel Robinson

CSC304 - COBOL

\today

\vspace{0.5in}

First appearing in 1959, COBOL has been an enduring force in the IT and Business World. 
Since that time, only 12\% of the Fortune 500 corporations that existed during the 1950s have survived to today.
Since the industry respected TIOBE rankings were introduced in 1989 to guage favorability of programming languages, COBOL has maintained a position amongst the top 30 languages.
Only the juggernauts C and C++ can boast the same credentials.

As Steve Dickens of IBM's LinuxONE project states:
\begin{quote}
    ``COBOL  is  60-years  young.   The  language  that  powers  the  mainframes  that  run  the  world  is as relevant today as it was in the 1960s.  With the presence of new digital pressures, the main-frame and COBOL are back at the forefront for the modern developer enabling innovation andtransformation.''
\end{quote}

Critical business systems are written using established technologies like COBOL. 
Newer, popular languages and technologies do not have the long history to compare.
United Life Insurance Company has utilized COBOL, according to Jim Veglahn:
``United Life Insurance Company built its core life and annuity policy systems on COBOL.''

Current market analysts indicate there is currently a shift from replacing old systems to simply modernizing them.
The pace of change in the modern world leaves no time to completely rebuild core business applications.
This is just another testament that COBOL fulfills the business's needs without being replaced by a newer technology.

Modern COBOL is designed to run on any system configuration.
The language works with all of the hot ``buzzword'' technologies today, natively, such as Docker, PostgreSQL, and Amazon Web Services.
Through this approach, COBOL remains a tool readily available for business application needs.

The design of the language achieves a number of goals. 
Its type-rich language allows data to be described accurately and precisely.
This leads to a validity that ensures it meets standards of the organization, its partner, and industry compliance requirements.
COBOL features industry leading arithmetic precision at 38 decimal digits, in a world where calculations cannot be a point of compromise.
Data manipulation is made fast and easy, with speedy data access comparable or better to relational database management systems.
Support for a wide variety of file formats are built in, with manipulation and reporting tools are a primary focus of the language.

COBOL is designed with accessibility anywhere in mind, meaning code written on one platform can be used on another.
Programmers and business analysts have widely different use cases for their machines, but COBOL runs on both the same.
It also features availability for platform or hardware specific optimizations, allowing for vast performance gains when every second counts.

Perhaps one of the main selling points of COBOL today is its ease of use and reading.
It is readily apparent to anyone reading what lines of code are achieving.
This makes it simple to understand and maintain, and more importantly, to learn.
Ease of learning is directly related to the availability of new developers for these businesses.


\end{document}
