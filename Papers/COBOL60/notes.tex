\documentclass[12pt]{article}

\usepackage[english]{babel}

\usepackage{amsmath}

\usepackage{graphicx}
\usepackage{tabularx}
\usepackage{multicol}
\usepackage{enumitem}

% Fancy Header
\usepackage{fancyhdr}
\renewcommand{\footrulewidth}{0.4pt}
\pagestyle{fancy}
\fancyhf{}
\chead{CSC 304 - COBOL}
\lfoot{CALU Fall 2021}
\rfoot{RDK}

% Geometry 
\usepackage{geometry}
\geometry{letterpaper, left=15mm, top=20mm, right=15mm, bottom=20mm}

% Add vertical spacing to tables
\renewcommand{\arraystretch}{2}

% Macros
\newcommand{\definition}[1]{\underline{\textbf{#1}}}

\usepackage{setspace}
\usepackage{etoolbox}
\AtBeginEnvironment{quote}{\singlespace\vspace{-\topsep}\small}
\AtEndEnvironment{quote}{\vspace{-\topsep}\endsinglespace}

% Begin Document
\begin{document}


\section{Origins}

\begin{itemize}

    \item First appearing in 1959, COBOL has been an enduring force in the IT and Business world
    \item Only 12\% of the Fortune 500 from the 1950s have survived to today
    \item COBOL has remained in the top 30 of the TIOBE rankings since they began in 1989. Only C and C++ can boast the same.
    \item Estimates are \$1.7 trillion is wasted on failed projects yearly
    \item ``COBOL is 60-years young. The language that powers the mainframes that run the world is as relevant today as it was in the 1960s. With the presence of new digital pressures, the mainframe and COBOL are back at the forefront for the modern developer enabling innovation and transformation'' - Steven Dickens IBM LinuxONE


\end{itemize}



\section{Applications}

\begin{itemize}

    \item Critical business systems are written using established technologies, not necessarily the new, popular technologies of the day.
    \item ``\textbf{United Life Insurance Company} built its core life and annuity policy systems on COBOL.'' - Jim Veglahn, ULIC

    \item Market analysts indicate a shift from replacement of systems to simple modernization of the systems, with half of respondents indicating projects starting within two years.
    \item The pace of change in the modern world leaves no time for replacements, just upgrades.
    \item The majority of today's ``buzzword'' technologies integrate with COBOL, such as AWS, Docker, and PostgreSQL.

\end{itemize}



\section{Design}

\begin{itemize}

    \item Modern COBOL is designed to run on any system configuration.
    \item \textbf{Robustness and validity}: COBOL's type-rich language allows data to be desscribed accurately with explicit scope and limits. This richness means you can meet your coporate coding standards, ensuring consistency and accuracy across your organization and third parties, including partners and industry specific compliance requirements.
    \item \textbf{Numeric arithmetic accuracy}: COBOL boats arithmetic to 38 decimal digits. The accuraccy of calculations cannot be a point of compromise, and powers the financial powerhouses of the world.
    \item \textbf{Strong data manipulation}:
    \begin{itemize}
        \item Faster data access than any RDBMS
        \item Support for many data file formats
        \item Data manipulation and reporting built into the language
    \end{itemize}
    \item \textbf{Performance}: COBOL can be built for specific hardware and platforms, allowing optimizations unavailable to generic systems and solutions.
    \item \textbf{Accessibility}: COBOL code is portable across platforms, allowing access wherever it's needed
    \item \textbf{Readability}:
    \begin{itemize}
        \item COBOL is designed to allow the reader to know at a glance what the code is trying to achieve
        \item Ease of entry to learning and thus jobs
    \end{itemize}


\end{itemize}

\end{document}